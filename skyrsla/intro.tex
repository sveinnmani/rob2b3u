\section{Inngangur}
Eftir að hafa komið með nokkrar hugmyndir þá síuðum við þær út og ákváðum að okkar langaði að búa til eftirlitsvélmenni því okkur fannst við eiga góða möguleika að búa til einfalt vélmenni sem gæti sinnt þessu hlutverki en þegar grunnvirknin er komin þá væri margir möguleikar á viðbætum sem mundi flækja verkefnið og gefa okkur nóg að gera alla  önnina.
Hvað er eftirlitsvélmenni? Eftirlitvélmenni í grunnin sér um að fara yfir eitthvað tiltekið svæði og tekur upp það sem það fer framhjá. Þetta þjónar sama tilgangi og eftirlitmyndavél en hefur þó þann kost að vera á hjólum og getur því fylgst með stærra svæði og einnig að það er hægt að nota vélmennið við mun fleiri aðstæður en hefðbundna eftirlitsmyndavél t.d. að nálgast, með því að handstýra, og skoða svæði þar sem er hættulegt eða annars ómögulegt fyrir manneskjur að komast að. Ókostir þess eru að það er töluvert flóknara að byggja vélmennið frekar en bara hefðbundna eftirlitsmyndavél og að það þarf að hlaða batterýið af og til á vélmenninu. 
Grunnvirknin á vélmenninu er ekki rosalega flókin og þurfum við aðeins nokkra parta til þess að fá hana í gang. Fyrst þurfum við að smíða grind ásamt festingum til að halda öllu uppi síðan myndavél til að taka upp það sem verður í vegi á vélmenninu eftir það þurfum við tvö hjól, sex öxla, sex gíra og tvo mótora sem sjá um það að koma vélmenninu af stað og í lokin þá þurfum við Tölvu og breadbord til að tengja og forrita alla íhlutina en við kusum að notast við Rasberry pi Tölvu. Síðan þurfum við líka leið til að birta myndina sem myndavélin tekur upp en við getum annaðhvort valið að geyma hana á einhverskonar drifi, að líkindum flash og geta yfirfarið myndbandið eftir á en þá þurfum við flashkubb sem er ágætlega stór til að halda utan um myndefnið. Hin leiðin er að streyma myndina beint á einhvern skjá en þá þurfum við að net tengja vélmennið okkar. Við höfum gert allt fyrir utan myndavélapartinn í þessu verkefni áður og ættum því ekki að vera lengi á fá einfalt vélmenni til að virka sem er gott því við munum alltaf enda með eitthvað nothæft í höndunum og getum bara bætt við það eftir því sem líður á tíman.
Viðbætunar eru ýmislegar og mjög misjafnt hversu flóknar þær eru. Fyrstu hlutirnir sem við myndum bæta við væru eflaust encoders til að geta mælt nákvæmlega hversu langt vélmennið fer og því gætum við búið til nákvæmari og flóknari leið sem vélmennið mundi fara, fyrir þetta mundum við nota tvo Vex encoders. Síðan gætum við bætt við þeim möguleika að manneskja gæti tekið við stýrinu á vélmenninu og stjórnað hvert það fer og hvað það skoðar, þetta mundum við líklegast leysa með þvi að búa til app í síma sem tengist við vélmennið annaðhvort með bluetooth eða í gegnum netið. Við getum bætt við Sonar sensor sem að mundi  gera vélmenninu kleyft að skynja og forðast aðskotahluti sem verða í vegi þess. Einnig gætum við forritað vélmennið þannig að sendi frá sér viðvörun eða einfaldlega taki bara upp þegar það skynjar eitthvað í kringum sig sem á ekki að vera þar. Þetta eru bara þær hugmyndir sem við höfum fengið sem komið er og eflaust margar viðbætur sem er hægt að útfæra ef tími gefst.
Að lokum þá finnst okkur sú hugmynd að við getum verið komnir með nothæft tæki á litlum tíma en síðan betrum bætt það eins mikið og við höfum tíma til gera þetta að skemmtilegu en samt krefjandi og áhugaverðu verkefni.

\begin{figure}[h]
\end{figure}